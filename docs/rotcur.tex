% -*- tex -*-

\documentclass[12pt]{article}

\begin{document}

\section{Rotation Curves (mom1)}

For convenience we've parameterized our input model rotation curves. We use the notation $r = R/R_0$, where $R_0$
is a radius scale factor.  All rotation curves have the property they are linear near the center, and
either flatten off, or drop at larger radii.

\noindent
The first one is an unrealistic exponentially (but see Greisen's eq.(2))
$$
V = V_0 ( 1 - e^{-r} )         \eqno(1)
$$
Here is a family of curves defined by the exponent $m$ (for $m -> \infty$ the linear+flat toy model is reproduced):
$$
V_m = V_0 {   {r}  \over  { (1+r^m)^{1/m} } }     \eqno(2)
$$
For $m=2$ this is the familiar logarithmic potential (BT, pp.45, eq. 2.54 and eq. 3.77) 
$$
    \Phi = {1\over 2} V_0^2
                     \ln{ \left( R_0^2 + R^2 \right) }  \eqno(2a)
$$
Greisen's paper also mentions the very old Brandt curves eq. (4) (using our notation):
$$
V_m = V_o {   r \over { (1/3 + 2/3 r^m)^{3/(2m)}   }    }     \eqno(3)
$$
where $m$ is the ``Brandt index''. Note that Brandt curves at large radii have no dark matter, i.e. $v \propto r^{-1/2}$
Levy's paper uses the universal Persic rotation curve profiles.
                     

\section{Density Profiles (mom0)}

The gas density profile is independent of the rotation curve, so these need to be parameterized as well.
Currently we have a constant and exponential dropoff.

\section{Velocity Dispersion(mom2)}

We normally assume the velocity dispersion of the gas to be constant as function of radius, but in order
to study any possible degeneracy, we could relax this and e.g. assume a larger velocity dispersion in the
central bulge regions.  We can use the same scaling radius ($R_0$ for this).

\section{Beam Smearing Correction (BSC)}

Begeman's 1989 BSC algorithm uses an implicit relationship between the Observed velocity, $v$, and the Model velocity, $V$,
which we rewrite here as follows:
$$
   v = V + \delta
$$
where the BSC term $\delta$ is given by
$$   
\delta = {b^2 \over n} ( {\partial{V}\over\partial{x}} {\partial{N}\over\partial{x}} + {\partial{V}\over\partial{y}} {\partial{N}\over\partial{y}}
 + N ({\partial^2{V}\over\partial{x^2}} + {\partial^2{V}\over\partial{y^2}}) / 2)
$$
or our shorthand notation
$$   
     \delta = {b^2 \over n} ( V_x N_x + V_y N_y + N (V_{xx} + V_{yy}) / 2)
$$   
which contains a 1st and 2nd order derivative term.

We now discuss a few common cases where the BSC can be evaluated analytically:

\subsection{Constant Density}
For the simplified case where the density is constant, 
the 1st order terms drop out and with $n=N$ we are left with:
$$
    \delta = {b^2\over 2} ( V_{xx} + V_{yy}) = {b^2\over 2} \nabla V 
$$
which looks like the Inhomogeneous Helmholtz equation with wavenumber $\sqrt{2}/b$.

For a flat rotation curve (and probably all) this term will have a $3\$phi$ type
pattern.
    

    
\subsection{Linear rotation curve}
It is also easy to see now that for a linearly rising rotation curve
$$    
V(x,y) = \Omega x
$$
and with a constant density the BSC term is 0.

In the case where the density is not constant, a linearly rising rotation curve follows
$$
   \delta =  b^2   \Omega  {N_x \over n}
$$
   
\subsection{Flat rotation curve}

For a flat rotatation curve the velocity field only depends on the position angle
$$
V(r,\phi) = V(\phi) = V_0 \sqrt{1-{\sin^2\phi \over \cos^2i}} \sin i
$$
and in a constant density disk this simplifies to
$$
\delta =  {b^2\over 2} \nabla V =  {b^2\over 2} {1\over r^2}   {\partial^2{V}\over\partial{\phi^2}}
$$
or
$$
\delta =  {-\alpha} { b^2\over 2} {1\over r^2}
  {  { (1-\alpha \sin^2\phi)(\cos^2\phi-\sin^2\phi) + \alpha \sin^2\phi \cos^2\phi}   \over { (1-\alpha \sin^2\phi)^{3/2}} }
$$

\section{Pragmatic Fitting}

Using the family of models in eq.(2) we have three parameters
($R_0,V_0,m$). For given beam size $B$ this will affect mostly $R_0$
and $m$, but not much, if any, in $V_0$. So, working in normalized
units where $R_0 = V_0 = 1$, this means that for given model with
index $m$ the smooth version of the rotation curve can be fit with
$r_1,m_1$, for a given $B$. We call these the B curves.


\section{Notation}

$N,V,S$ are the model density,velocity and velocity dispersion, and
$n,v,s$ those in the smoothed observed velocity field.  We align the
major axis of the tilted disk along the X axis (PA=90) where this
matters in the equations above.

  
\end{document}                     
