% -*- tex -*-

\documentclass[12pt]{article}

\begin{document}

\section{Rotation Curves (mom1)}

For convenience we've parameterized our input model rotation curves. We use the notation $r = R/R_0$, where $R_0$
is a radius scale factor.  All rotation curves have the property they are linear near the center, and
either flatten off, or drop at larger radii.

\noindent
The first one is an unrealistic exponentially

$$
V = V_0 ( 1 - e^{-r} )         \eqno(1)
$$


\noindent
Here is a family of curves defined by the exponent $m$ (for $m -> \infty$ the linear+flat toy model is reproduced):

$$
V_m = V_0 {   {r}  \over  { (1+r^m)^{1/m} } }     \eqno(2)
$$


\noindent
For $m=2$ this is the familiar logarithmic potential (BT, pp.45, eq. 2.54 and eq. 3.77) 
$$
    \Phi = {1\over 2} V_0^2
                     \ln{ \left( R_0^2 + R^2 \right) }  \eqno(2a)
$$

\section{Density Profiles (mom0)}

The gas density profile is independent of the rotation curve, so these need to be parameterized as well.
Currently we have a constant and exponential dropoff.

\section{Velocity Dispersion(mom2)}

We normally assume the velocity dispersion of the gas to be constant as function of radius, but in order
to study any possible degeneracy, we could relax this and e.g. assume a larger velocity dispersion in the
central bulge regions.  We can use the same scaling radius ($R_0$ for this). 


\end{document}                     
