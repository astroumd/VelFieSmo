% -*- tex -*-

\documentclass[12pt]{article}

\begin{document}

\section*{Beam Smearing of Galaxy Velocity Fields}

% \bigskip\noindent
Version: PJT 9-jul-2019


\subsection*{Notation}

$N,V,S$ are the model density,velocity and velocity dispersion, and
$n,v,s$ those in the smoothed observed velocity field.  We align the
major axis of the tilted disk along the X axis (PA=90) where this
matters in the equations below.  The radius and velocity
if often normalized to $R_0$ and $V_0$.



\section{V(R): Rotation Curve (mom1)}

For convenience we've parameterized our input model rotation curves. We use the notation $r = R/R_0$, where $R_0$
is a radius scale factor.  All rotation curves have the property they are linear near the center, and
either flatten off, or drop at larger radii.

\noindent
The first one is an unrealistic ``fixed shape'' exponentially (but see Greisen's eq.(2))
$$
V = V_0 ( 1 - e^{-r} )         \eqno(1)
$$
Here is a family of curves defined by a shape parameter via
the exponent $m$ (for $m -> \infty$ the linear+flat toy model is reproduced):
$$
V_m = V_0 {   {r}  \over  { (1+r^m)^{1/m} } }     \eqno(2)
$$
For $m=2$ this is the familiar logarithmic potential (BT, pp.45, eq. 2.54 and eq. 3.77) 
$$
    \Phi = {1\over 2} V_0^2
                     \ln{ \left( R_0^2 + R^2 \right) }  \eqno(2a)
$$
Greisen's paper also mentions the very old Brandt curves eq. (4) (using our notation):
$$
V_m = V_o {   r \over { (1/3 + 2/3 r^m)^{3/(2m)}   }    }     \eqno(3)
$$
where $m$ is the ``Brandt index''. Note that Brandt curves at large radii have no dark matter, i.e. $v \propto r^{-1/2}$.
Levy's paper uses the universal Persic rotation curve profiles, with the $L_B/L_{B*}$ shape parameter.
                     

\section{N(R): Density Profile (mom0)}

The gas density profile is independent of the rotation curve, so these need to be parameterized as well.
Currently we have a exponential dropoff option (for constant take a very large $R_e$):
$$
   N(R) = N_0 e^{-R/R_e}     \eqno(4)
$$
where roughly $R_e/R_{to} \approx 1.93 \pm 0.96$. Note the turnover radius $R_{to}$ and the
beforementioned $R_0$ are not the same, their ratio $R_{to}/R_0 > 1$ but will approach 1 as
$m$ gets very large.
   
   

\section{S(R): Velocity Dispersion (mom2)}

We normally assume the velocity dispersion of the gas to be constant as function of radius, but in order
to study any possible degeneracy, we could relax this and e.g. assume a larger velocity dispersion in the
central bulge regions.  We can use the same scaling radius ($R_0$ for this).   In the current modeling
code we keep it constant.

\section{Beam Smearing Correction (BSC): $\delta$}

Begeman's 1989 BSC algorithm uses an implicit relationship between the Observed velocity, $v$, and the Model velocity, $V$, which we rewrite here as follows:
$$
   v = V + \delta     \eqno(5)
$$
where the BSC term $\delta$ is given by
$$   
\delta = {b^2 \over n} ( {\partial{V}\over\partial{x}} {\partial{N}\over\partial{x}} + {\partial{V}\over\partial{y}} {\partial{N}\over\partial{y}}
 + N ({\partial^2{V}\over\partial{x^2}} + {\partial^2{V}\over\partial{y^2}}) / 2)       \eqno(6a)
$$
or our shorthand notation
$$   
     \delta = {b^2 \over n} ( V_x N_x + V_y N_y + N (V_{xx} + V_{yy}) / 2)        \eqno(6b)
$$   
which contains a 1st and 2nd order derivative term.  Here $b$ the size of the beam (FWHM/2.355)
in terms of the pixel size, a dimensionless number of order 1-2. Begeman (priv.comm.) tried the
Lucy method to compute $V$ from $v$ but did not succeed. A little better were the
gaussian profile fits using the Schwarz method (need ref).

We now discuss a few common cases where $\delta$ can be evaluated analytically:

\subsection{Constant Density}

For the simplified case where the density is constant, 
the 1st order terms drop out and with $n=N$ we are left with:
$$
    \delta = {b^2\over 2} ( V_{xx} + V_{yy}) = {b^2\over 2} \nabla V      \eqno(7)
xs$$
which looks like the Inhomogeneous Helmholtz equation with wavenumber $\sqrt{2}/b$.

For a flat rotation curve (and probably all) this term will have a $3\phi$ type
pattern.
    

    
\subsection{Linear rotation curve}

It is also easy to see now that for a linearly rising rotation curve
$$    
V(x,y) = \Omega x     \eqno(8)
$$
and with a constant density the BSC term is 0.

In the case where the density is not constant, a linearly rising rotation curve follows
$$
   \delta =  b^2   \Omega  {N_x \over n}     \eqno(9)
$$
   
\subsection{Flat rotation curve}

For a flat rotatation curve the velocity field only depends on the position angle
$$
V(r,\phi) = V(\phi) = V_0 \sqrt{1-{\sin^2\phi \over \cos^2i}} \sin i          \eqno(10)
$$
and in a constant density disk this simplifies to
$$
\delta =  {b^2\over 2} \nabla V =  {b^2\over 2} {1\over r^2}   {\partial^2{V}\over\partial{\phi^2}}        \eqno(11)
$$ 
or if my math was done right
$$
\delta =  {-\alpha} { b^2\over 2} {1\over r^2}
  {  { (1-\alpha \sin^2\phi)(\cos^2\phi-\sin^2\phi) + \alpha \sin^2\phi \cos^2\phi}   \over { (1-\alpha \sin^2\phi)^{3/2}} }        \eqno(12)
$$

\section{Pragmatic Fitting}


The explicit nature of eq. (5) means we cannot compute the model from the observations,
except under special circumstances, e.g. assuming a flat rotation curve [sic] and/or
constant density disk. However, for a parameterized rotation curve a model can be derived:

Using the family of models in eq.(2) we have three model parameters
($V_0,R_0,m$). For given beam size $B$ this will result in a new
rotation curve, but generally with a different shape,
will affect mostly $R_0$
and $m$, but not much, if any, $V_0$.   We call the fitted
values ($V_1,R_1,m_1$). We can create a grid of such models,
and retrieve the input model
($V_0,R_0,m$) from this.  [degeneracy TBD]

\section{Random Notes}

\subsection{exp vs. core}

Comparing Greisen's ``exp'' curve with the ``core-m'' family, the m=2 curve crosses over around $R/R_0 \approx 3$. Doing a fit
of the exp to core-m has the usual mixed results:   shape=(V0=102.43,R0=154.07,m=2.05),
mod=(102.9,146.7,1.91), slit=(102.7,146.7,1.91) and ring=(103.9,136.3,1.72), which certainly visually agrees with $m \approx 1.8-2.2$.
See also {\tt model\_rot\_curve.py}.

\subsection{kinematic inclination}

In our fits we fix the inclination to the morphological inclination. The kinematic inclination is typically a few degrees lower
than the morphological one, see {\tt smooth\_inc.py}. Either way, a residual velocity field always shows the familiar $3\phi$ symmetry.

\subsection{rotcur fitting}

Fitting using rotcur often over-estimates. Near the center this is understandable, since there are few points, but in some cases
in the outer parts this is also visible. Hence we started using rotcurshape, but given the kinematic inclination effect, the slit
and shape will then give different results.

\end{document}                     
